\includegraphics[width=2.36458in,height=0.875in]{input_mediafulllatex/media/image1.jpeg}

\begin{longtable}[]{@{}
  >{\raggedright\arraybackslash}p{(\linewidth - 8\tabcolsep) * \real{0.1964}}
  >{\raggedright\arraybackslash}p{(\linewidth - 8\tabcolsep) * \real{0.2685}}
  >{\raggedright\arraybackslash}p{(\linewidth - 8\tabcolsep) * \real{0.1139}}
  >{\raggedright\arraybackslash}p{(\linewidth - 8\tabcolsep) * \real{0.1139}}
  >{\raggedright\arraybackslash}p{(\linewidth - 8\tabcolsep) * \real{0.3074}}@{}}
\toprule\noalign{}
\begin{minipage}[b]{\linewidth}\centering
\textbf{SUBJECTS}
\end{minipage} & \begin{minipage}[b]{\linewidth}\centering
\textbf{XII STD}
\end{minipage} &
\multicolumn{2}{>{\centering\arraybackslash}p{(\linewidth - 8\tabcolsep) * \real{0.2278} + 2\tabcolsep}}{%
\begin{minipage}[b]{\linewidth}\centering
\textbf{WT - 1}
\end{minipage}} & \begin{minipage}[b]{\linewidth}\centering
\textbf{NEET QP}
\end{minipage} \\
\midrule\noalign{}
\endhead
\bottomrule\noalign{}
\endlastfoot
\textbf{PHYSICS} &
\multicolumn{4}{>{\raggedright\arraybackslash}p{(\linewidth - 8\tabcolsep) * \real{0.8036} + 6\tabcolsep}@{}}{%
\textbf{ELECTROSTATICS (S1-S4)}} \\
\textbf{CHEMISTRY} &
\multicolumn{4}{>{\raggedright\arraybackslash}p{(\linewidth - 8\tabcolsep) * \real{0.8036} + 6\tabcolsep}@{}}{%
\textbf{SOLUTIONS (S1-S3)}} \\
\textbf{BIOLOGY} &
\multicolumn{4}{>{\raggedright\arraybackslash}p{(\linewidth - 8\tabcolsep) * \real{0.8036} + 6\tabcolsep}@{}}{%
\textbf{SEXUAL REPRODUCTION IN FLOWERING PLANTS (S1-S3)}} \\
\emph{\textbf{TOTAL MARKS -- 720}} &
\multicolumn{2}{>{\centering\arraybackslash}p{(\linewidth - 8\tabcolsep) * \real{0.3824} + 2\tabcolsep}}{%
\emph{\textbf{DURATION -- 3 HRS}}} & & \\
\emph{\textbf{EACH QUESTION CARRIES 4 MARKS. (-1 MARK) FOR WRONG
ANSWER.}} & & & & \\
\end{longtable}

\textbf{\ul{PHYSICS}}

\begin{enumerate}
\def\labelenumi{\arabic{enumi}.}
\item
  Among two discs A and B, first have radius 10 cm and charge 10 cm and
  10\textsuperscript{-6} C and second have radius 30 cm and charge
  10\textsuperscript{-5} C. When they are touched, charge on both
  q\textsubscript{A} and q\textsubscript{B} respectively will, be:
\end{enumerate}

a) q\textsubscript{A} = 2.75 μC, q\textsubscript{B} = 3.15 μC b)
q\textsubscript{A} = 1.09 μC, q\textsubscript{B} = 1.53 μC

c) q\textsubscript{A} = q\textsubscript{B} = 5.5μC d) None of these

\begin{enumerate}
\def\labelenumi{\arabic{enumi}.}
\setcounter{enumi}{1}
\item
  A negatively charged object X is repelled by another charged object Y.
  However an object Z is attracted to object Y. Which of the following
  is the most possibility for the object Z?\\
  a) positively charged only b) negatively charged only\\
  c) neutral or positively charged d) neutral or negatively charged
\item
  Four objects W, X, Y and Z, each with charge +q are held fixed at four
  points of a square of side d as shown in the figure. Objects X and Z
  are on the midpoints of the sides of the square. The electrostatic
  force exerted by object W on object X is F. Then the magnitude of the
  force exerted by object W on Z is\\
  \includegraphics[width=1.92835in,height=1.88677in]{input_mediafulllatex/media/image2.png}\\
  a) \(\frac{F}{7}\) b) \(\frac{F}{5}\) c) \(\frac{F}{3}\) d)
  \(\frac{F}{2}\) e) \(\frac{f}{2}\) g)
  \(\frac{2}{2} = \frac{3}{4} - \frac{4}{6}\)
\item
  An uncharged metal object M is insulated from its surroundings. A
  positively metal sphere S is then brought near to M. Which diagram
  best illustrates the resultant distributions of charge on S and M?
\end{enumerate}

a)
\includegraphics[width=2.15333in,height=0.62526in]{input_mediafulllatex/media/image3.png}
b)
\includegraphics[width=2.22667in,height=0.6171in]{input_mediafulllatex/media/image4.png}

c)
\includegraphics[width=2.18in,height=0.62497in]{input_mediafulllatex/media/image5.png}
d)
\includegraphics[width=2.29333in,height=0.66991in]{input_mediafulllatex/media/image6.png}

\begin{enumerate}
\def\labelenumi{\arabic{enumi}.}
\setcounter{enumi}{4}
\item
  \textbf{Statement I:} Electric charge is additive in nature\\
  \textbf{Statement II:} The total charge of a system is the algebraic
  sum of individual charges\\
  a) Both Statements are true, and Statement II is correct explanation
  of Statement I\\
  b) Both Statements are true, but Statement II is not correct
  explanation of Statement I\\
  c) Statement I is true Statement II is false\\
  d) Statement I is false Statement II is true
\item
  When a body is charged by induction, then the body\\
  a) becomes neutral b) does not lose any charge\\
  c) loses whole of the charge on it d) loses part of the charge on it
\item
  Assume that each of copper atom has one free electron and there is 1
  mg of copper. Given that atomic weight of copper = 63.5 and Avogadro
  number = 6.02 × 10\textsuperscript{23}. What is the charge possessed
  by these free electrons?
\end{enumerate}

a) 1.52 C b) 1.76 C c) 4.76 C d) 1.25 C

\begin{enumerate}
\def\labelenumi{\arabic{enumi}.}
\setcounter{enumi}{7}
\item
  When a comb rubbed with dry hair attracts pieces of paper. This is
  because the\\
  a) comb polarizes the piece of paper\\
  b) comb induces a net dipole moment opposite to the direction of
  field\\
  c) electric field due to the comb is uniform\\
  d) comb induces a net dipole moment perpendicular to the direction of
  field
\item
  Two copper balls, each weighing 10g, are kept in air 10cm apart. If
  one electron from every 10\textsuperscript{6} atoms is transferred
  from one ball to the other, the coulomb force between them is (atomic
  weight of copper is 63.5)
\end{enumerate}

a) 2.0 × 10\textsuperscript{10}N b) 2.0 × 10\textsuperscript{4}N c) 2.0
× 10\textsuperscript{8}N d) 2.0 × 10\textsuperscript{6} N

\begin{enumerate}
\def\labelenumi{\arabic{enumi}.}
\setcounter{enumi}{9}
\item
  \textbf{Assertion (A):} Coulomb force is the dominating force in the
  universe\\
  \textbf{Reason (R):} Coulomb force is weaker than the gravitational
  force\\
  a) Both Assertion and Reason are true and Reason is correct
  explanation of Assertion\\
  b) Both Assertion and Reason are true but Reason is not the correct
  explanation of Assertion\\
  c) If Assertion is true and Reason is false\\
  d) If both Assertion and Reason is false
\item
  Two identical balls each have a mass of 10g. What charges should these
  balls be given so that their interaction equalizes the force of
  universal gravitation acting between them? The radii of the balls may
  be ignored in comparison to distance between them.
\end{enumerate}

a) 6.34 × 10\textsuperscript{-11}C b) 8.57 × 10\textsuperscript{-11} C
c) 6.34 ×~10\textsuperscript{−13}C d) 8.57 × 10\textsuperscript{-13} C

\begin{enumerate}
\def\labelenumi{\arabic{enumi}.}
\setcounter{enumi}{11}
\item
  Two point charges +3 µC and +8 µC repel each other with a force of 40
  N. If a charge of -5 µC is added to each of them, then the force
  between them will become
\end{enumerate}

a) -- 10N b) +10 N c) +20N d) -- 20 N

\begin{enumerate}
\def\labelenumi{\arabic{enumi}.}
\setcounter{enumi}{12}
\item
  Calculate the ratio of electrostatic to gravitational force between
  two electrons placed at certain distance in air. Given that
  m\textsubscript{e} = 9.1 × 10\textsuperscript{-31} kg, e = 1.6 ×
  10\textsuperscript{-19} C and G = 6.6 × 10\textsuperscript{-11} N
  m\textsuperscript{2} kg\textsuperscript{-2}
\end{enumerate}

a) 8.4 × 10\textsuperscript{42} b) 3.2 × 10\textsuperscript{41} c) 4.2 ×
10\textsuperscript{42} d) 1.2 × 10\textsuperscript{42}

\begin{enumerate}
\def\labelenumi{\arabic{enumi}.}
\setcounter{enumi}{13}
\item
  Three charges each equal to 2μC are placed at the corners of an
  equilateral triangle. If force between any two charges is 2F, then the
  net force on either will be\\
  a) \(2\sqrt{3}F\) b) \(\sqrt{2}F\) c) 3F d) F/3
\item
  Two charges +4e and +e are at a distance x apart. At what distance a
  charge q must be placed from charge +e so that it is in equilibrium?
\end{enumerate}

a) x/2 b) 2x/3 c) x/3 d) x/6

\begin{enumerate}
\def\labelenumi{\arabic{enumi}.}
\setcounter{enumi}{15}
\item
  Three charges each of magnitude q are placed at the corners of an
  equilateral triangle. The electrostatic force on the charge placed at
  the centre is (each side of triangle is L)
\end{enumerate}

a) Zero b) \(\frac{1}{4\pi\varepsilon_{0}}\frac{q^{2}}{L^{2}}\) c)
\(\frac{1}{4\pi\varepsilon_{0}}\frac{3q^{2}}{L^{2}}\) d)
\(\frac{1}{12\pi\varepsilon_{0}}\frac{q^{2}}{L^{2}}\)

\begin{enumerate}
\def\labelenumi{\arabic{enumi}.}
\setcounter{enumi}{16}
\item
  Match the scenarios in Column A with the corresponding descriptions in
  Column B
\end{enumerate}

\begin{longtable}[]{@{}
  >{\raggedright\arraybackslash}p{(\linewidth - 2\tabcolsep) * \real{0.5004}}
  >{\raggedright\arraybackslash}p{(\linewidth - 2\tabcolsep) * \real{0.4996}}@{}}
\toprule\noalign{}
\begin{minipage}[b]{\linewidth}\centering
\textbf{Column A}
\end{minipage} & \begin{minipage}[b]{\linewidth}\centering
\textbf{Column B}
\end{minipage} \\
\midrule\noalign{}
\endhead
\bottomrule\noalign{}
\endlastfoot
A) Equal charges placed at the corners of a square & i) Forces on the
charges are balanced along diagonals \\
B) Charges of different magnitudes placed on a line & ii) Force depends
on magnitude and sign of charges \\
C) Net force on a charge in a system & iii) Vector sum of all individual
forces \\
D) Zero net force on a test charge & iv) Charges symmetrically placed
around the test charge \\
\end{longtable}

a) (A-i), (B-ii), (C-iii), (D-iv) b) (A-i), (B-ii), (C-iv), (D-iii)\\
c) (A-i), (B-iii), (C-ii), (D-iv) d) (A-iv), (B-iii), (C-i), (D-ii)

\begin{enumerate}
\def\labelenumi{\arabic{enumi}.}
\setcounter{enumi}{17}
\item
  \textbf{Assertion (A):} Megaspore mother cell undergoes meiosis to
  produce four haploid gametes of which all are functional\\
  \textbf{Reason (R):} Megaspore mother cell is 2n, mitosis gives
  haploid structure.
\end{enumerate}

a) Assertion and reason are true and reason is the correct explanation
of assertion

b) Assertion and reason are true but reason is not the correct
explanation of assertion

c) Assertion is true but reason is false

d) Both assertion and reason are false
\item
\begin{enumerate}
\def\labelenumi{\arabic{enumi}.}
\setcounter{enumi}{147}
\end{enumerate}
