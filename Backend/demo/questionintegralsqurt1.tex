\begin{enumerate}
\def\labelenumi{\arabic{enumi}.}
\item
  Among two discs A and B, first have radius 10 cm and charge 10 cm and
  10\textsuperscript{-6} C and second have radius 30 cm and charge
  10\textsuperscript{-5} C. When they are touched, charge on both
  q\textsubscript{A} and q\textsubscript{B} respectively will, be:
\end{enumerate}

a) q\textsubscript{A} = 2.75 μC, q\textsubscript{B} = 3.15 μC b)
q\textsubscript{A} = 1.09 μC, q\textsubscript{B} = 1.53 μC

c) q\textsubscript{A} = q\textsubscript{B} = 5.5μC d) None of these

\begin{enumerate}
\def\labelenumi{\arabic{enumi}.}
\setcounter{enumi}{1}
\item
  A negatively charged object X is repelled by another charged object Y.
  However an object Z is attracted to object Y. Which of the following
  is the most possibility for the object Z?\\
  a) positively charged only b) negatively charged only\\
  c) neutral or positively charged d) neutral or negatively charged
\item
  Four objects W, X, Y and Z, each with charge +q are held fixed at four
  points of a square of side d as shown in the figure. Objects X and Z
  are on the midpoints of the sides of the square. The electrostatic
  force exerted by object W on object X is F. Then the magnitude of the
  force exerted by object W on Z is\\
  a) \(\frac{F}{7}\) b) \(\frac{F}{5}\) c) \(\frac{F}{3}\) d)
  \(\frac{F}{2}\) e) \(\frac{f}{2}\) g)
  \(\frac{2}{2} = \frac{3}{4} - \frac{4}{6}\)
\item
  \textbf{Statement I:} Electric charge is additive in nature\\
  \textbf{Statement II:} The total charge of a system is the algebraic
  sum of individual charges\\
  a) Both Statements are true, and Statement II is correct explanation
  of Statement I\\
  b) Both Statements are true, but Statement II is not correct
  explanation of Statement I\\
  c) Statement I is true Statement II is false\\
  d) Statement I is false Statement II is true
\item
  When a body is charged by induction, then the body\\
  a) becomes neutral b) does not lose any charge\\
  c) loses whole of the charge on it d) loses part of the charge on it
\item
  Assume that each of copper atom has one free electron and there is 1
  mg of copper. Given that atomic weight of copper = 63.5 and Avogadro
  number = 6.02 × 10\textsuperscript{23}. What is the charge possessed
  by these free electrons?
\end{enumerate}

a) 1.52 C b) 1.76 C c) 4.76 C d) 1.25 C

\begin{enumerate}
\def\labelenumi{\arabic{enumi}.}
\setcounter{enumi}{6}
\item
  When a comb rubbed with dry hair attracts pieces of paper. This is
  because the\\
  a) comb polarizes the piece of paper\\
  b) comb induces a net dipole moment opposite to the direction of
  field\\
  c) electric field due to the comb is uniform\\
  d) comb induces a net dipole moment perpendicular to the direction of
  field
\item
  Two copper balls, each weighing 10g, are kept in air 10cm apart. If
  one electron from every 10\textsuperscript{6} atoms is transferred
  from one ball to the other, the coulomb force between them is (atomic
  weight of copper is 63.5)
\end{enumerate}

a) 2.0 × 10\textsuperscript{10}N b) 2.0 × 10\textsuperscript{4}N c) 2.0
× 10\textsuperscript{8}N d) 2.0 × 10\textsuperscript{6} N

\begin{enumerate}
\def\labelenumi{\arabic{enumi}.}
\setcounter{enumi}{8}
\item
  \textbf{Assertion (A):} Coulomb force is the dominating force in the
  universe\\
  \textbf{Reason (R):} Coulomb force is weaker than the gravitational
  force\\
  a) Both Assertion and Reason are true and Reason is correct
  explanation of Assertion\\
  b) Both Assertion and Reason are true but Reason is not the correct
  explanation of Assertion\\
  c) If Assertion is true and Reason is false\\
  d) If both Assertion and Reason is false
\item
  Two identical balls each have a mass of 10g. What charges should these
  balls be given so that their interaction equalizes the force of
  universal gravitation acting between them? The radii of the balls may
  be ignored in comparison to distance between them.
\end{enumerate}

a) 6.34 × 10\textsuperscript{-11}C b) 8.57 × 10\textsuperscript{-11} C
c) 6.34 ×~10\textsuperscript{−13}C d) 8.57 × 10\textsuperscript{-13} C

\begin{enumerate}
\def\labelenumi{\arabic{enumi}.}
\setcounter{enumi}{10}
\item
  Two point charges +3 µC and +8 µC repel each other with a force of 40
  N. If a charge of -5 µC is added to each of them, then the force
  between them will become
\end{enumerate}

a) -- 10N b) +10 N c) +20N d) -- 20 N

\begin{enumerate}
\def\labelenumi{\arabic{enumi}.}
\setcounter{enumi}{11}
\item
  Calculate the ratio of electrostatic to gravitational force between
  two electrons placed at certain distance in air. Given that
  m\textsubscript{e} = 9.1 × 10\textsuperscript{-31} kg, e = 1.6 ×
  10\textsuperscript{-19} C and G = 6.6 × 10\textsuperscript{-11} N
  m\textsuperscript{2} kg\textsuperscript{-2}
\end{enumerate}

a) 8.4 × 10\textsuperscript{42} b) 3.2 × 10\textsuperscript{41} c) 4.2 ×
10\textsuperscript{42} d) 1.2 × 10\textsuperscript{42}

\begin{enumerate}
\def\labelenumi{\arabic{enumi}.}
\setcounter{enumi}{12}
\item
  Three charges each equal to 2μC are placed at the corners of an
  equilateral triangle. If force between any two charges is 2F, then the
  net force on either will be\\
  a) \(2\sqrt{3}F\) b) \(\sqrt{2}F\) c) 3F d) F/3
\item
  Two charges +4e and +e are at a distance x apart. At what distance a
  charge q must be placed from charge +e so that it is in equilibrium?
\end{enumerate}

a) x/2 b) 2x/3 c) x/3 d) x/6

\begin{enumerate}
\def\labelenumi{\arabic{enumi}.}
\setcounter{enumi}{14}
\item
  Three charges each of magnitude q are placed at the corners of an
  equilateral triangle. The electrostatic force on the charge placed at
  the centre is (each side of triangle is L)
\end{enumerate}

a) Zero b) \(\frac{1}{4\pi\varepsilon_{0}}\frac{q^{2}}{L^{2}}\) c)
\(\frac{1}{4\pi\varepsilon_{0}}\frac{3q^{2}}{L^{2}}\) d)
\(\frac{1}{12\pi\varepsilon_{0}}\frac{q^{2}}{L^{2}}\)

\begin{enumerate}
\def\labelenumi{\arabic{enumi}.}
\setcounter{enumi}{15}
\item
  Match the scenarios in Column A with the corresponding descriptions in
  Column B
\end{enumerate}

a) (A-i), (B-ii), (C-iii), (D-iv) b) (A-i), (B-ii), (C-iv), (D-iii)\\
c) (A-i), (B-iii), (C-ii), (D-iv) d) (A-iv), (B-iii), (C-i), (D-ii)

\begin{enumerate}
\def\labelenumi{\arabic{enumi}.}
\setcounter{enumi}{16}
\item
  A charge q is placed at the centre of the line joining two equal
  charges Q. The system of the three charges will be in equilibrium, if
  q is equal to
\end{enumerate}

a) \(- \frac{Q}{2}\) b) \(- \frac{Q}{4}\) c) \(+ \frac{Q}{4}\) d)
\(+ \frac{Q}{2}\)

\begin{enumerate}
\def\labelenumi{\arabic{enumi}.}
\setcounter{enumi}{17}
\item
  Three charges +q, +2q and 4q are connected by strings as shown in the
  figure and are in equilibrium. What is ratio of tension in the strings
  AB and BC?
\end{enumerate}

a) 1 : 2 b) 1 : 3 c) 2 : 1 d) 3 : 1

\begin{enumerate}
\def\labelenumi{\arabic{enumi}.}
\setcounter{enumi}{18}
\item
  Infinite charges of magnitude q each are lying at x = 1, 2, 4,
  8\ldots{} meter on x- axis. The value of intensity of electric field
  at point x = 0 due to these charges will be
\end{enumerate}

a) 12 × 10\textsuperscript{9} q N/C b) zero c) 6 ×
10\textsuperscript{9}qN/C d) 4 × 10\textsuperscript{9}q N/C

\begin{enumerate}
\def\labelenumi{\arabic{enumi}.}
\setcounter{enumi}{19}
\item
  Two concentric rings, one of radius R and total charge +Q and the
  second of radius R and total charge \(- \sqrt{8}Q\), lie in x-y place
  (i.e., z = plane). The common centre of rights lies at origin and the
  common axis coincides with z -- axis. The charge is uniformly
  distributed on both the rings. At what distance from origin is the net
  electric field on z -- axis zero?
\end{enumerate}

a) \(\frac{R}{2}\) b)\(\sqrt[2]{av + 234}\) c)
\(\frac{R}{\sqrt{2a + 2b}}\) d) \(\sqrt{a^{2} + b^{2}}\)

\begin{enumerate}
\def\labelenumi{\arabic{enumi}.}
\setcounter{enumi}{20}
\item
  How many electrons should be removed from a coin of mass 1.6 g, so
  that it may float in an electric field of intensity
  10\textsuperscript{9} NC\textsuperscript{-1} directed upward. (Take g
  = 10 m/s\textsuperscript{2})
\end{enumerate}

a) \(\sum_{}^{}{3/2}\) b) \(\overset{\mathrm{\Delta}}{\rightarrow}45\)
c) \(\int_{1\backslash 2}^{ex}4\) d) \(\int_{as4}^{45}{\sum_{}^{}5}\)

\begin{enumerate}
\def\labelenumi{\arabic{enumi}.}
\setcounter{enumi}{21}
\item
  An electron of mass m, initially at rest, moves through a certain
  distance in a uniform electric field in time t. A proton of mass M,
  also initially at rest, takes time T to move through an equal distance
  in this uniform electric field. Neglecting the effect of gravity, the
  ratio \(\frac{T}{t}\) is nearly equal to\\
  a) \(\frac{2M}{m}\) b) \(\sqrt{\frac{m}{M}}\) c)
  \(\sqrt{\frac{M}{m}}\) d) \(\frac{M}{m}\)
\item
  In the electric field shown in figure, the electric field lines on the
  left have twice the separation as that between those on the right. If
  the magnitude of the field at point A is 40 NC\textsuperscript{-1},
  calculate the magnitude of electric field at the point B.
\end{enumerate}

a) 15 NC\textsuperscript{-1} b) 20 NC\textsuperscript{-1} c) 25
NC\textsuperscript{-1} d) 30 NC\textsuperscript{-1}

\begin{enumerate}
\def\labelenumi{\arabic{enumi}.}
\setcounter{enumi}{23}
\item
  An electron is released with a velocity of 5 × 10\textsuperscript{6}
  ms\textsuperscript{-1} in an electric field of 10\textsuperscript{3}
  NC\textsuperscript{-1} which has been applied so as to oppose its
  motion. How much time could it take before it is brought to rest?
\end{enumerate}

a) 2.8 × 10\textsuperscript{-8} s b) 1.8 × 10\textsuperscript{-8} s c)
6.4 × 10\textsuperscript{-8} s d) 4.2 × 10\textsuperscript{-8} s

\begin{enumerate}
\def\labelenumi{\arabic{enumi}.}
\setcounter{enumi}{24}
\item
  Under the influence of the coulomb field to charge +Q, a charge --q is
  moving around it in an elliptical orbit. Find out the correct
  statement(s)
\end{enumerate}

a) The angular momentum of the charge --q is constant

b) The linear momentum of the charge --q is constant

c) The angular velocity of the charge --q is constant

d) The linear speed of the charge --q is constant

\begin{enumerate}
\def\labelenumi{\arabic{enumi}.}
\setcounter{enumi}{25}
\item
  Two identical conducting spheres carrying different charges attract
  each other with a force F when placed in air medium at a distance `d'
  apart. The spheres are brought into contact and then taken to their
  original positions. Now the two spheres repel each other with a force
  whose magnitude is equal to that of the initial attractive force. The
  ratio between initial charges on the spheres is
\end{enumerate}

a) \(- \left( 3 + \sqrt{8} \right)only\) b) \(- 3 + \sqrt{8}only\)

c) \(+ \left( 3 + \sqrt{8} \right)or\left( + 3 - \sqrt{8} \right)\) d)
\(+ \sqrt{3}\)

\begin{enumerate}
\def\labelenumi{\arabic{enumi}.}
\setcounter{enumi}{26}
\item
  The electric field that can balance a charged particle of mass 4.8 x
  10\textsuperscript{-27} kg is: (Given that the charge on the particle
  is 1.6 x 10\textsuperscript{-19} C)\\
  a) 19.6 ×10\textsuperscript{-8} NC\textsuperscript{-1} b) 29.4 ×
  10\textsuperscript{-8} NC\textsuperscript{-1} c) 29.7 ×
  10\textsuperscript{8} NC\textsuperscript{-1} d) 19.6 ×
  10\textsuperscript{-7} NC\textsuperscript{-1}
\item
  An infinite dielectric sheet having charge density σ has a hole of
  radius R in it. An electron is released on the axis of the hole at a
  distance \(\sqrt{3}R\) from the centre. The speed with which it
  crosses the centre of the hole\\
  a) \(\sqrt{\frac{\sigma eR}{2m\varepsilon_{0}}}\) b)
  \(\sqrt{\frac{\sigma eR}{m\varepsilon_{0}}}\) c)
  \(\sqrt{\frac{2\sigma eR}{m\varepsilon_{0}}}\) d) None of these
\item
  Two identical positive point charges each of value Q are fixed at the
  points (a, 1) and (-a, 0) on the\\
  x-axis. A particle of mass m and carrying charge -q is released from
  rest at the point \(P\left( 0,\sqrt{3}a \right)\)on the\\
  y-axis. The velocity υ\textsubscript{0} of the particle when it passes
  through the origin O υ\textsubscript{0}. Then\\
  a) \(\upsilon_{0} = \sqrt{\frac{Qq}{\pi\varepsilon_{0}ma}}\) b)
  \(\upsilon_{0} = \sqrt{\frac{Qq}{2\pi\varepsilon_{0}ma}}\) c)
  \(\upsilon_{0} = \sqrt{\frac{Qq}{4\pi\varepsilon_{0}ma}}\) d)
  \(\upsilon_{0} = \sqrt{\frac{Qq}{8\pi\varepsilon_{0}ma}}\)
\item
  Two identical charges repel each other with a force equal to 10 mg wt
  when they are 0.6 m apart in air
\end{enumerate}

(g = 10ms\textsuperscript{-2}). The value of each charge is

a) 2 mC b) 2 × 10\textsuperscript{-7}C c) 2 nC d) 2 µC

\begin{enumerate}
\def\labelenumi{\arabic{enumi}.}
\setcounter{enumi}{30}
\item
  Electric lines of force about n negative point charge are\\
  a) circular anticlockwise b) circular clockwise c) radial, inwards d)
  radial, outwards
\item
  Identify the wrong statement in the following. Coulomb's law correctly
  describes the electric force that
\end{enumerate}

a) Binds the electrons of an atom to its nucleus

b) Binds the protons and neutrons in the nucleus of an atom

c) Binds atoms together to form molecules

d) Binds atoms and molecules together to form solids

\begin{enumerate}
\def\labelenumi{\arabic{enumi}.}
\setcounter{enumi}{32}
\item
  If a linear isotropic dielectric is placed in an electric field of
  strength E, then the polarization P is\\
  a) independent of E b) inversely proportional to E\\
  c) directly proportional to \(\sqrt{E}\) d) directly proportional to E
\item
  In a uniform electric field, a charge +q having negligible mass is
  released at a point. Which of the following statements are correct?\\
  I. Velocity increases wit time\\
  II. A force acts on it in the direction of electric field\\
  III. Its momentum changes with time\\
  a) I and II b) II and III c) I and III d) I, II and III
\item
  A positively charged rod is brought near an uncharged conductor. If
  the rod is then suddenly withdrawn, the charge left on the conductor
  will be\\
  a) Positive b) Negative c) Zero d) cannot say
\item
  Three charges -q\textsubscript{1}, +q\textsubscript{2} and
  -q\textsubscript{3} are placed as shown in the figure. The x-component
  of the force on -q\textsubscript{1} is proportional to\\
  a) \(\frac{q_{2}}{b^{2}} - \frac{q_{3}}{a^{2}}\cos\theta\) b)
  \(\frac{q_{2}}{b^{2}} + \frac{q_{3}}{a^{2}}\sin\theta\) c)
  \(\frac{q_{2}}{b^{2}} + \frac{q_{3}}{a^{2}}\cos\theta\) d)
  \(\frac{q_{2}}{b^{2}} - \frac{q_{3}}{a^{2}}\sin\theta\)
\item
  The magnitude of the average electric field normally pres­ent in the
  atmosphere just above the surface of the Earth is about 150 N/C,
  directed inward towards the centre of the Earth. This gives the total
  net surface charge car­ried by the Earth to be (Given
  ε\textsubscript{0} = 8.85 x 10\textsuperscript{-12}
  C\textsuperscript{2}/N m\textsuperscript{2}, R\textsubscript{E} = 6.37
  x 10\textsuperscript{6} m)\\
  a) +670 kC b) -670 kC c) -680 kC d) +680 kC
\item
  \textbf{Assertion (A):} If there exists coulomb attraction between two
  bodies, both of them must be charged
\end{enumerate}

\textbf{Reason (R):} In coulomb attraction two bodies are oppositely
charged\\
a) Both Assertion and Reason are true and Reason is correct explanation
of Assertion\\
b) Both Assertion and Reason are true but Reason is not the correct
explanation of Assertion\\
c) Assertion is true but Reason is false\\
d) Assertion is false but Reason is true

\begin{enumerate}
\def\labelenumi{\arabic{enumi}.}
\setcounter{enumi}{38}
\item
  When a body is earth connected, electrons from the earth flow into the
  body. This means the body is
\end{enumerate}

a) Uncharged b) charged positively

c) Charged negatively d) an insulator

\begin{enumerate}
\def\labelenumi{\arabic{enumi}.}
\setcounter{enumi}{39}
\item
  For a uniformly charged ring of radius R, the electric field on its
  axis has the largest magnitude at a distance h from its centre. Then
  value of h is\\
  a) \(\frac{R}{\sqrt{5}}\) b) \(\frac{R}{\sqrt{2}}\) c) R d)
  \(R\sqrt{2}\)
\item
  In the figure, charge q is placed at origin O. When the charge q is
  displaced from its position the electric field at point P changes\\
  \strut \\
  a) at the same time when q is displaced b) at a time after
  \(\frac{OP}{c}\)where c is the speed of light\\
  c) at a time after \(\frac{OP\cos\theta}{c}\) d) at a time after
  \(\frac{OP\sin\theta}{c}\)
\item
  Match the following
\end{enumerate}

a) (A-ii), (B-iii), (C-i), (D-iv) b) (A-i), (B-iii), (C-i), (D-iv)\\
c) (A-iii), (B-i), (C-ii), (D-iv) d) (A-iii), (B-ii), (C-i), (D-iv)

\begin{enumerate}
\def\labelenumi{\arabic{enumi}.}
\setcounter{enumi}{42}
\item
  Two large vertical and parallel metal plates having a separation of
  1cm are connected to a DC voltage source of potential difference X. A
  proton is released at rest midway between the two plates. It is found
  to move at 45˚ to the vertical just after release. Then X is nearly
\end{enumerate}

a) 1 × 10\textsuperscript{-5}V b) 1 × 10\textsuperscript{-7}V c) 1 ×
10\textsuperscript{-9} V d) 1 × 10\textsuperscript{-10}V

\begin{enumerate}
\def\labelenumi{\arabic{enumi}.}
\setcounter{enumi}{43}
\item
  The mean free path of electrons in a metal is 4 ×
  10\textsuperscript{-8} m. The electric field which can give on an
  average 2eV energy to an electron in the metal will be in units of V/m
\end{enumerate}

a) 8 × 10\textsuperscript{7} b) 5 × 10\textsuperscript{-11} c) 8 ×
10\textsuperscript{-11} d) 5 × 10\textsuperscript{7}

\begin{enumerate}
\def\labelenumi{\arabic{enumi}.}
\setcounter{enumi}{45}
\item
  The density of a 2.03 M solution of acetic acid (molecular mass 60) in
  water is 1.017 g/mL. Calculate the molality of the solution\\
  a) 2.26 b) 3.28 c) 2.3 d) 4.2
\item
  How much water is needed to dilute 10 ml of 10N hydrochloric acid to
  make it exactly decinormal\\
  (0.1 N)?\\
  a) 990 ml b) 1000 ml c) 1010 ml d) 100 ml
\item
  The density of a 3 M sodium thiosulphate solution
  (Na\textsubscript{2}S\textsubscript{2}O\textsubscript{3}) is 1.25
  g/mL. Calculate the percentage by mass of sodium thiosulphate\\
  a) 38.3 b) 35.6 c) 37.9 d) 40.5
\item
  A solution of known concentration is known as\\
  a) Molar solution b) Normal solution\\
  c) Mole solution d) Standard solution
\item
  The mole fraction of CH\textsubscript{3}OH in an aqueous solution is
  0.02 and its density is 0.994 g cm\textsuperscript{-3}. Determine its
  molarity\\
  a) 1.08 b) 2.3 c) 3.28 d) 1.23
\item
  When the volume of the solution is doubled, the following becomes
  exactly half\\
  a) Molality b) Mole -- fraction c) Molarity d) weight percent
\item
  Calculate the concentration of NaOH solution in g/mL, which has the
  same normality as that of a solution of HCl of concentration 0.04
  g/mL.\\
  a) 0.258 b) 0.0235 c) 0.652 d) 0.0438
\item
  The solution having lowest molar concentration is\\
  a) 1.0N HCl b) 0.4N H\textsubscript{2}SO\textsubscript{4} c) 0.1N
  Na\textsubscript{2}CO\textsubscript{3} d) 1N NaOH
\item
  How many Na\textsuperscript{+} ions are present in 50 mL of, a 0.5 M
  solution of NaCl?\\
  a) 1.055 ×10\textsuperscript{23} b) 1.055 ×10\textsuperscript{22} c)
  3.055 ×10\textsuperscript{22} d) 4.055 ×10\textsuperscript{22}
\item
  Density of a 2.05 M solution of acetic acid in water is 1.02 g/mL. The
  molality of the solution is\\
  a) 1.14 mol kg\textsuperscript{-1} b) 3.28 mol kg\textsuperscript{-1}
  c) 2.28 mol kg\textsuperscript{-1} d) 0.44 mol kg\textsuperscript{-1}
\item
  Which of the following has no units?\\
  a) Molarity b) Normality c) Molarity d) Mole fraction
\item
  The vapour pressure of pure benzene at 88°C is 957 mm and that of
  toluene at the same temperature is 379.5 mm. Calculate the composition
  of benzene-toluene mixture boiling at 88°C\\
  a) x\textsubscript{benzene} = 0.66 ; x\textsubscript{toluene} = 0.34
  b) x\textsubscript{benzene} = 0.34 ; x\textsubscript{toluene} = 0.66\\
  c) x\textsubscript{benzene} = x\textsubscript{toluene} = 0.5 d)
  x\textsubscript{benzene} = 0.75 ; x\textsubscript{toluene} = 0.25
\item
  The vapour pressures of ethanol and methanol are 44.5 mm and 88.7 mm
  Hg respectively. An ideal solution is formed at the same temperature
  by mixing 60 g of ethanol with 40 g of methanol. Calculate the total
  vapour pressure of the solution\\
  a) 85.36 b) 56.23 c) 66.13 d) 42.56
\item
  An aqueous solution of Methyl alcohol contains 48g of alcohol. The
  mole fraction of alcohol is 0.6. The weight of water in it is\\
  a) 27 g b) 2.7 g c) 18g d) 1.8 g
\item
  The mass of glucose that would be dissolved in 50 g of water in order
  to produce the same lowering of vapour pressure as is produced by
  dissolving 1 g of urea in the same quantity of water is\\
  a) 1g b) 3g c) 6g d) 18g
\item
  Henry's law constant for CO\textsubscript{2} in water is 1.678 x
  10\textsuperscript{8} pa at 298 K calculate the quantity of
  CO\textsubscript{2} in 500 ml of soda water when packed under 2.5 atm
  pressure at 298K\\
  a) 0.0084g b) 0.0184g c) 1.848g d) 8.4g
\item
  At 25°C, the total pressure of an ideal solution obtained by mixing 3
  mole of A and 2 mole of B, is\\
  184 torr. What is the vapour pressure (in torr) of pure B at the same
  temperature (Vapour pressure of pure A at 25°C is 200 torr)?\\
  a) 180 b) 160 c) 16 d) 100
\item
  Vapour pressure is the pressure exerted by vapours\\
  a) In equilibrium with liquid b) In any condition\\
  c) In an open system d) in atmospheric condition
\item
  100 mL of liquid A and 25 mL of liquid B are mixed to form a solution
  of volume 125 mL. Then the solution is\\
  a) ideal b) non-ideal with positive deviation\\
  c) non-ideal with negative deviation d) cannot be predicted
\item
  What volume of a 0.8M solution contains 100 millimoles of the
  solute?\\
  a) 100 ml b) 125 ml c) 500 ml d) 62.5 ml
\item
  A semi molar solution is the one, which contains\\
  a) One mole solute in 2 litres b) 2 moles solute in 2 litres\\
  c) 0.1 mole solute in 1 litres d) 0.2 moles solute in 2 litres
\item
  The vapour pressure of a certain pure liquid A at 298 K is 40 m bar.
  When a solution of B is prepared in A at the same temperature, the
  vapour pressure is found to be 32 m bar. The mole fraction of A in the
  solution is:\\
  a) 0.5 b) 0.2 c) 0.1 d) 0.8
\item
  A solution of 36\% water and 64\% acetaldehyde
  (CH\textsubscript{3}CHO) by mass. Mole fraction of acetaldehyde is\\
  a) 0.42 b) 0.2 c) 4.2 d) 2.1
\item
  Vapour pressure of pure \(A\left( p_{A}^{0} \right) = 100\) mm Hg\\
  Vapour pressure of pure \(B\left( p_{B}^{0} \right) = 150\) mm Hg\\
  2 mole of liquid A and 3 mole of liquid B are mixed to form an ideal
  solution. The vapour pressure of solution will be\\
  a) 135 mm b) 130 mm c) 140 mm d) 145 mm
\item
  The weight in grams of KCl (Mol.wt = 74.5) in 100 ml of a 0.1 M KCl
  solution is\\
  a) 74.5 b) 2024 c) 0.745 d) 0.0745
\item
  138 grams of ethyl alcohol is mixed with 72 grams of water. The ratio
  of mole fraction of alcohol to water is\\
  a) 3:4 b) 1:2 c) 1:4 d) 1:1
\item
  The density (in g mL\textsuperscript{-1}) of a 3.60M sulphuric acid
  solution that s 29\% H\textsubscript{2}SO\textsubscript{4} (molar
  mass=98 g mol\textsuperscript{-1}) by mass, will be\\
  a) 1.45 b) 1.64 c) 1.88 d) 1.22
\item
  Four gases like H\textsubscript{2}, He, CH\textsubscript{4} and
  CO\textsubscript{2} have Henry's constant values (K\textsubscript{H})
  are 69.16, 144.97, 0.413 and 1.67. The gas which is more soluble in
  liquid is\\
  a) He b) CH\textsubscript{4} c) H\textsubscript{2} d)
  CO\textsubscript{2}
\item
  Mole fraction of the component A in vapour phase is x\textsubscript{1}
  and the mole fraction of component A in liquid mixture is
  x\textsubscript{2}, then ( \(p_{A}^{0}\) = vapour pressure of pure
  \(A;p_{B}^{0}\) vapour pressure of pure B), the total vapour pressure
  of liquid mixture is\\
  a) \(p_{A}^{0}\frac{x_{2}}{x_{1}}\) b)
  \(p_{A}^{0}\frac{x_{1}}{x_{2}}\) c) \(p_{B}^{0}\frac{x_{1}}{x_{2}}\)
  d) \(p_{B}^{0}\frac{x_{2}}{x_{1}}\)
\item
  If 0.01 mole of solute is present in 500 ml of solution, its molarity
  is\\
  a) 0.01 M b) 0.005M c) 0.02M d) 0.1 M
\item
  \textbf{Assertion:} A mixture of cyclohexane and ethanol shows --ve
  deviation from Raoult's law.\\
  \textbf{Reason:} Cyclohexane reduces the intermolecular attraction
  between ethanol molecules.\\
  a) Assertion and reason are true and reason is the correct explanation
  of assertion\\
  b) Assertion and reason are true but reason is not the correct
  explanation of assertion\\
  c) Assertion is true but reason is false\\
  d) Assertion is false but reason is true
\item
  \textbf{Assertion (A):} Addition of solvent to a solution always
  lowers the vapour pressure\\
  \textbf{Reason (R) :} The increase in relative surface area given rise
  to an increase in vapour pressure for a given solution\\
  a) Assertion and reason are true and reason is the correct explanation
  of assertion\\
  b) Assertion and reason are true but reason is not the correct
  explanation of assertion\\
  c) Assertion is true but reason is false\\
  d) Assertion is false but reason is true
\item
  Heptane and octane form ideal solution. At 373 K, the vapour pressures
  of the two liquids are 105.2 kPa and 46.8 kPa respectively. What will
  be the vapour pressure, in bar, of a mixture of 25 g of heptane and 35
  g of octane?\\
  a) 0.7308 b) 0.6523 c) 0.5263 d) 0.6528
\item
  Calculate the quantity of sodium carbonate (anhydrous) required to
  prepare 250 ml 0.1M solution\\
  a) 2.65 grams b) 4.95 grams c) 6.25 grams d) none of these
\item
  1 litre solution containing 490 g of sulphuric acid is diluted to 10
  litre with water. What is the normality of the resulting solution?\\
  a) 0.5 N b) 1.0 N c) 5.0 N d) 10.0 N
\item
  Maltose is converted to `A' by Maltase. The mole fraction of `A' in
  10\% (w/w) aq. Solution is approximately\\
  a) 0.18 b) 0.989 c) 0.1 d) 0.017
\item
  250 mL of a Na\textsubscript{2}CO\textsubscript{3} solution contains
  2.65 g of Na\textsubscript{2}CO\textsubscript{3}. 10 mL of this
  solution is added to x mL of water to obtain 0.001 M
  Na\textsubscript{2}CO\textsubscript{3} solution. The value of x
  is....\\
  (Molecular mass of Na\textsubscript{2}CO\textsubscript{3} = 106 amu)\\
  a) 1000 b) 990 c) 9990 d)90
\item
  100 ml 0.2M NaOH is exactly neutralised by a mixture of which of the
  following?\\
  a) 100 ml of 0.1 M HCl + 100 ml of 0.1 M
  H\textsubscript{2}SO\textsubscript{4\\
  }b) 100 ml of 0.1 M HCl + 50 ml of 0.1 M
  H\textsubscript{2}SO\textsubscript{4\\
  }c) 50 ml of 0.1 M HCl + 50 ml of 0.1 M
  H\textsubscript{2}SO\textsubscript{4\\
  }d) 50 ml of 0.1 M HCl + 100 ml of 0.1 M
  H\textsubscript{2}SO\textsubscript{4}
\item
  CO\textsubscript{(g)} is dissolved in H\textsubscript{2}O at 30˚C and
  0.020 atm. Henry's law constant for this system is 6.20 ×
  10\textsuperscript{4} atm. Thus, mole fraction of
  CO\textsubscript{(g)} is\\
  a) 1.72 × 10\textsuperscript{-7} b) 3.22 × 10\textsuperscript{-7} c)
  0.99 d) 0.01
\item
  An unopened soda has an aqueous concentration of CO\textsubscript{2}
  at 25˚C equal to 0.0408 molal. Thus, pressure of CO\textsubscript{2}
  gas in the can is (K\textsubscript{H} = 0.034 mol/kg bar)\\
  a) 0.671 bar b) 1.49 bar c) 1.39 bar d) 1.71 bar
\item
  The volumes of two-HCl solutions A (0.5 N) and B (0.1 N) to be mixed
  for preparing 2 L of 0.2 N HC1 are\\
  a) 0.5 L of A + 1.5 L of B b) 1.5 L of A + 0.5 L of B\\
  c) 1 L of A + 1 L of B d) 0.75 L of A + 1.25 L of B
\item
  Two bottles A and B contain 2 M and 2m aqueous solutions of sulphuric
  acid respectively, then\\
  a) A is more concentrated than B b) B is more concentrated than A\\
  c) Conc. of A and B are equal d) It is impossible to compare the
  concentrations
\item
  An aqueous solution containing 28\% by mass of a liquid A (mol. mass -
  140) has a vapour pressure of 160 mm at 37°C. Find the vapour pressure
  of the pure liquid A (The vapour pressure of water at 37°C is 150
  mm)\\
  a) 562.8 b) 360.15 c) 450.6 d) 503.4
\item
  An aqueous solution of glucose is 10\% in strength. The volume in
  which 2 g mole of it is dissolved will be\\
  a) 18 litre b) 3.6 litre c) 0.9 litre d) 1.8 lite
\item
  The hardness of water sample containing 0.002 mol of magnesium
  sulphate dissolved in a litre of water is expressed as\\
  a) 20 ppm b) 200 ppm c) 2000 ppm d) 120 ppm
\end{enumerate}

\begin{enumerate}
\def\labelenumi{\arabic{enumi}.}
\setcounter{enumi}{90}
\item
  A typical angiosperm anther is\\
  a) Monolobed, monothecous and bisporangiate\\
  b) Bilobed, monothecous and tetrasporangiate\\
  c) Bilobed, dithecous and tetrasporangiate\\
  d) Bilobed, dithecous and bisporangiate
\item
  Read the following statements and identify the wrong statements.\\
  A. Stamen length and number vary amongst the flowers of the same
  species only.\\
  B. Angiosperm anthers are bilobed, with two theca in each.\\
  C. The theca is often separated by a longitudinal groove.\\
  D. The anther has four microsporangia, one in each lobe.\\
  E. The microsporangia grow into pollen sacs.\\
  a) B, C and E b) A, C and D c) A and D only d) A and B only
\item
  Match the following\\
  \textbf{Structure Shape}\\
  A. Anther 1. Spindle shaped\\
  B. Microsporangium 2. Spherical shaped\\
  C. Pollen grain 3. Tetragonal (four sided)\\
  D. Generative cell 4. Near circular in outline
\item
  Arrangement of four wall layers in microsporangium from inside to
  outside is as follows\\
  a) Epidermis, endothecium, tapetum and middle layers\\
  b) Epidermis, middle layers, endothecium and tapetum\\
  c) Epidermis, endothecium, middle layers and tapetum\\
  d) Tapetum, middle layers, endothecium and epidermis
\item
  Diameter of the pollen grain is generally\\
  a) 5 -- 10 μm b) 10 -- 50 μm c) 20 -- 50 μm d) 25 - 50 μm
\item
  When the pollen grain is mature it contains two cells, the vegetative
  cell and generative cell. The vegetative cell\\
  A. Is bigger B. Spindle shaped\\
  C. Has abundant food reserve D. Has large irregularly shaped nucleus\\
  a) A, B and C b) A, C and D\\
  c) A, B, C and D d) B, C and D
\item
  An anther having four microsporocytes shall produce \_\_\_\_ pollen
  grains.\\
  a) 24 b) 12 c) 8 d) 16
\item
  An ovule generally has a single embryo sac formed from a megaspore
  through\\
  a) Reduction division\\
  b) Mitotic divisions\\
  c) Mitotic division followed by meiotic division\\
  d) Meiotic division followed by mitotic division
\item
  Read the following statements and identify the wrong statements.\\
  A. The placenta is placed inside the locule, and the ovules develop
  from it.\\
  B. An ovary can have one or several ovules (papaya, watermelon, and
  orchids).\\
  C. Each ovule contains one or two protective envelopes known as
  integuments.\\
  D. The ovule is surrounded by integuments, with the exception of a
  small aperture called the chalaza near the tip. The micropylar end is
  located opposite the chalaza.\\
  E. Within the integuments lies a mass of cells known as the
  perisperm.\\
  a) B, D and E b) A, C and D c) B, C and E d) A, B and D
\item
  Match column I with column II, and choose the correct combination from
  the options given below.\\
  \textbf{Column I Column II\\
  }A. Male gametophyte 1. Ovule\\
  B. Female gametophyte 2. Locule\\
  C. Megasporangium 3. Pollen grain\\
  D. Ovarian cavity 4. Embryo sac
\item
  The following figure shows the\\
  \strut \\
  a) Multicarpellary syncarpous pistil of \emph{Papaver}\\
  b) Multicarpellary apocarpous gynoecium of \emph{Michelia}\\
  c) Pentacarpellary syncarpous gynoecium of the \emph{Hibiscus}\\
  d) Multicarpellary apocarpous gynoecium of the china rose
\item
  Read the following statements and find out the incorrect statements.\\
  A. Ovules generally differentiate single megaspore mother cell (MMC)
  in the chalazal region of the nucellus.\\
  B. The MMC undergoes reduction division and produces four
  megaspores.\\
  C. In a majority of angiosperms, one of the megaspores degenerate
  while the other three remains functional.\\
  D. The nucleus of the functional megaspore divides mitotically three
  times and form 2-nucleate, 4-nucleate and later 8-nucleate stages of
  the embryo sac.\\
  E. These mitotic divisions are strictly free nuclear, that is, nuclear
  division are immediately followed by cell wall formation.\\
  a) A, B and C b) B, C and D c) C, D and E d) A, C and E
\item
  Recognize the figure check out the correct matching\\
  a) a-nucellus, b-chalazal end, c-microspore dyad, d-microspore tetrad,
  e-megaspore mother cell\\
  b) a-megaspore mother cell, b-chalazal end, c-megaspore dyad,
  d-megaspore tetrad, e-nucellus\\
  c) a-megaspore mother cell, b-micropylar end, c-megaspore dyad,
  d-megaspore tetrad, e-nucellus\\
  d) a-nucellus, b-micropylar end, c-megaspore dyad, d-megaspore tetrad,
  e-megaspore mother cell
\item
  Embryo sac is monosporic when it develops from\\
  a) One of the four megaspores of a megaspore mother cell\\
  b) Three megaspores of megaspore tetrad\\
  c) Two functional megaspores\\
  d) The megaspore mother cell where meiosis has occurred but
  cytokinesis does not take place
\item
  Embryo sac develops from megaspore mother cell through\\
  a) 1 meiosis and 2 mitosis b) 1 meiosis and 3 mitosis\\
  c) 2 meiosis and 1 mitosis d) 2 meiosis and 2 mitosis
\item
  Read the following statements and identify the wrong statements.\\
  A. Plants use two abiotic (wind and water) and one biotic (animals)
  agents to pollinate.\\
  B. The majority of plants rely on abiotic agents for pollination.\\
  C. Only a small percentage of plants utilize biotic agents.\\
  D. Abiotic pollinators frequently use water to pollinate.\\
  E. Wind pollination is rare in flowering plants, affecting just about
  30 taxa, the majority of which are monocotyledons.\\
  a) A, B, C and D b) B, C, D and E c) A, C, D and E d) B and D only
\item
  Read the following statements and find out the incorrect statement\\
  a. Majority of flowering plants use a range of animals as pollinating
  agents.\\
  b. Bees, butterflies, flies, beetles, wasps, ants, moths, birds
  (sunbirds and hummingbirds) and bats are the common pollinating
  agents.\\
  c. Among the animals, insects, particularly bees, are the dominant
  biotic pollinating agents.\\
  d. Even larger animals such as some primates (lemurs), arboreal (tree
  dwelling) rodents, or even reptiles (gecko lizards and garden lizards)
  have also been reported as pollinators in some species.\\
  e. Often flowers of animal pollinated plants are specifically adapted
  for a particular species of animal.\\
  a) a and b b) b and c c) d and e d) None of the above
\item
  The flower height in \emph{Amorphophallus} is\\
  a) 6 feet b) 6 meter c) 6 cm d) 12 meter
\item
  Which of the following is an outbreeding device?\\
  A. If pollen release and stigma receptivity are not synchronized.\\
  B. If the anther and stigma are placed at different positions so that
  pollen cannot come in contact with the stigma of the same flower.\\
  C. Self-incompatibility which prevents self-pollen (from the same
  plant) from fertilizing the ovules by inhibiting pollen germination or
  pollen tube growth in the pistil.\\
  D. Production of the unisexual flower.\\
  a) A, B and C b) B, C and D c) A, C and D d) A, B, C and D
\item
  Recognize the figure and find out the correct matching\\
  \strut \\
  a) a-chasmogamous flowers, b-cleistogamous flowers\\
  b) a-cleistogamous flowers, b-chasmogamous flowers\\
  c) a-chamogamous flowers, b-dichogamous flowers\\
  d) a-dichogamous flowers, b-cleistogamous flowers
\item
  The given figure show the pollination by water in Vallisneria. Find
  out the correct matching.\\
  \strut \\
  a) a-female flower, b-male flower, c-female flower, d-stigma\\
  b) b-female flower, a-male flower, d-female flower, c-stigma\\
  c) a-female flower, b-male flower, d-female flower, c-stigma\\
  d) d-female flower, c-male flower, a-female flower, b-stigma
\item
  Pollination in lotus is carried out by\\
  a) Wind b) Water c) Insects d) All of the above
\item
  The occurrence of feathery stigma is noted in\\
  a) Pea b) Wheat/Jowar c) \emph{Datura} d) \emph{Caesalpinia}
\item
  Assertion: Each cell of sporogenous tissue is a potential pollen
  mother cell (PMC) or microspore mother cell\\
  Reason: Each cell of the sporogenous tissue is capable of giving rise
  to a microspore tetrad\\
  a) Both assertion and reason are true and the reason is the correct
  explanation of the assertion\\
  b) Both assertion and reason are true but reason is not the correct
  explanation of the assertion\\
  c) Assertion is true but reason is false\\
  d) Both assertion and reason are false
\item
  Assertion: Wind pollination requires that the pollen grains are light
  and non -- sticky\\
  Reason: Light pollen grains can be transported easily in wind
  currents\\
  a) Both assertion and reason are true and the reason is the correct
  explanation of the assertion\\
  b) Both assertion and reason are true but reason is not the correct
  explanation of the assertion\\
  c) Assertion is true but reason is false\\
  d) Both assertion and reason are false
\item
  In the diagram given, parts labelled as a, b, c, d, e and f are
  respectively identified as\\
  \strut \\
  a) Synergids, polar nuclei, central cell, antipodals, filiform
  apparatus and egg\\
  b) Polar nuclei, egg, antipodals, central cell, filiform apparatus and
  synergids\\
  c) Filiform apparatus, polar nuclei, egg, antipodals synergids and
  central cell\\
  d) Central cell, polar nuclei, filiform apparatus, antipodals,
  synergids and egg
\item
  Number of male gametes formed by 16 microspore mother cells is\\
  a) 128 b) 64 c) 32 d) 16
\item
  Identify the pair of wrong statements\\
  I. Intine of pollen grains is made up of sporopollenin,\\
  II. Pollen grains are well preserved as fossils because of the
  presence of sporopollenin\\
  III. Enzymes can degrade the organic material of the pollen grain
  exine\\
  IV. Sporopollenin can withstand high temperature, strong acids and
  alkali\\
  a) III, IV b) I, III c) I, II d) II, III
\item
  Cells present in mature male gametophyte of angiosperms is\\
  a) One b) Two c) Three d) Four
\item
  Match the following
\end{enumerate}

\textbf{Column - I Column - II}\\
A. Water pollination 1. Corn\\
B. Wind pollination 2. \emph{Yucca}\\
C. Insect pollination 3. \emph{Hydrilla}\\
\textbf{A B C}\\
a) 1 2 3\\
b) 3 1 2\\
c) 2 3 1\\
d) 3 2 1

\begin{enumerate}
\def\labelenumi{\arabic{enumi}.}
\setcounter{enumi}{120}
\item
  Male and female reproductive structures of the angiosperms are\\
  a) Carpel and pistil respectively b) Pistil and stamen respectively\\
  c) Gynoecium and androecium respectively d) Androecium and gynoecium
  respectively
\item
  The bilobed nature of an anther is very distinct in the\\
  a) Transverse section b) Longitudinal section\\
  c) Latitudinal section d) Both a and b
\item
  In the centre of each microsporangium, there is a group of compactly
  arranged homogenous cells called\\
  a) Tapetum b) Nucellus\\
  c) Sporogenous tissue d) Pollen grains
\item
  The microspores, as they are formed, are arranged in a cluster of four
  cells - the microspore tetrad. As the anthers mature and dehydrate,
  the microspores dissociate from each other and develop into\\
  a) Pollen grains b) Female gametophyte\\
  c) Male gametophyte d) Both a and c
\item
  Pollen grain has a prominent two layered wall. The inner wall\\
  a) Is made up of cellulose and pectin b) Is thin and continuous\\
  c) Is made up of sporopollenin d) Both a and b
\item
  Parthenium or carrot grass has become ubiquitous and causes pollen
  allergy. Parthenium came into India as a contaminant with imported\\
  a) Wheat b) Rice c) Carrot d) Rose
\item
  Which of the following statement about sporopollenin is wrong?\\
  a) Exine is formed of sporopollenin\\
  b) Sporopollenin is not degraded by any known enzyme\\
  c) Sporopollenin occurs in the area of germ pores only\\
  d) Sporopollenin is most resistant organic material
\item
  During formation of pollen grains, a microspore mother cell
  undergoes\\
  a) One meiotic division b) One mitotic division\\
  c) One meiotic and one mitotic division d) One meiotic and two mitotic
  divisions
\item
  The process of formation of microspores from pollen mother cell (PMC)
  through meiosis is called\\
  a) Microgametogenesis b) Microsporogenesis\\
  c) Megagametogenesis d) Megasporogenesis
\item
  In flowering plants, the male gametes are formed by\\
  a) Generative cell b) Uninucleate microspore\\
  c) Vegetative cell d) Pollen tube
\item
  Which one of the following statements is not true?\\
  a) Pollen grains of many species cause severe allergies\\
  b) Stored pollen in liquid nitrogen can be used in the crop breeding
  programmes\\
  c) Tapetum helps in the dehiscence of anther\\
  d) Exine of pollen grains is made up of sporopollenin
\item
  Proximal end of the filament of stamen is attached to the\\
  a) Placenta b) Thalamus or petal c) Anther d) Connective tissue
\item
  Largest cell of the ovule is\\
  a) Megaspore mother cell b) Antipodal cell\\
  c) Central cell d) Size of cells variable
\item
  In female gametophytes, the first haploid cell is
\end{enumerate}

a) Functional megaspore b) Microspore mother cell

c) Megaspore mother cell d) None of the above

\begin{enumerate}
\def\labelenumi{\arabic{enumi}.}
\setcounter{enumi}{134}
\item
  Which one produces embryo sac?\\
  a) Megaspore mother cell b) Megaspore\\
  c) Microspore d) Embryo cell
\item
  Which of the following statements is false?\\
  I. \emph{Vallisneria} and \emph{Hydrilla} are fresh water plants while
  sea-grasses (e.g. \emph{Zostera}) are marine plant.\\
  II. \emph{Vallisneria} is epihydrophilous while \emph{Zostera} is
  hypohydrophilous\\
  III. Pollination in water lily / Lotus (\emph{Nymphea}) and
  \emph{Eichhornia} (water hyacinth) takes place by insects\\
  IV. In majority of aquatic plants flowers emerge above the level of
  water and are pollinated by insects or wind\\
  V. In most of the water pollinated species, pollen grains are
  protected from wetting due to absence of mucilaginous covering\\
  VI. In hydrophilous plants pollen grains are spherical\\
  a) All b) V and VI c) None d) IV
\item
  Sequence of development of embyo sac is\\
  a) Archesporium → Megaspore → megasporpyte → Embryo sac\\
  b) Archesporium → Megaspore mother cell → Embryo sac → Megaspore\\
  c) Archesporium → Megaspore → Megaspore mother cell → Embryo sac\\
  d) Archeporium → Megaspore mother cell → Megaspore → Embryo sac
\item
  Identify the two cell stage of the pollen grain of the following
\end{enumerate}

a) A b) B c) C d) D

\begin{enumerate}
\def\labelenumi{\arabic{enumi}.}
\setcounter{enumi}{138}
\item
  Identify the following from the T.S. of anther\\
  a) A-Endothecium, B-Epidermis, C-Tapetum\\
  b) A-Epidermis, B-Endothecium, C-Tapetum\\
  c) A-Epidermis, B-Middle layers, C-Tapetum\\
  d) A-Epidermis, B-Middle layers, C-Endothecium
\item
  What does the following diagram represent?\\
  a) Female gametophyte b) Male gametophyte\\
  c) Egg d) Microspore mother cell
\item
  \textbf{Statement -A:} In over 60 per cent of angiosperms, pollen
  grains are shed at 3-celled stage\\
  \textbf{Statement -B:} \emph{Parthenium} or carrot grass came into
  India as a contaminant with imported wheat\\
  a) Statement A is correct but Statement B is incorrect\\
  b) Statement A is incorrect but Statement B is correct\\
  c) Both statements are correct\\
  d) Both statements are incorrect
\item
  What is the function of tassels in the corn cob?\\
  a) To disperse pollen grains b) To protect seeds\\
  c) To attract insects d) To trap pollen grains
\item
  \textbf{Assertion (A):} Attraction of flowers prevents all insects
  from damaging other parts of the plant\\
  \textbf{Reason (R):} Pollination carried out by insects is called
  Anemophily\\
  a) Assertion and reason are true and reason is the correct explanation
  of assertion
\end{enumerate}

b) Assertion and reason are true but reason is not the correct
explanation of assertion

c) Assertion is true but reason is false

d) Both assertion and reason are false

\begin{enumerate}
\def\labelenumi{\arabic{enumi}.}
\setcounter{enumi}{143}
\item
  Large, colourful, fragrant flowers with nectar are seen in\\
  a) bat pollinated plants b) wind pollinated plants\\
  c) insect pollinated plants d) bird pollinated plants
\item
  \textbf{Statement - A:} When there are more than one, the pistils may
  be fused together (syncarpous) or may be free (apocarpous)\\
  \textbf{Statement - B:} Inside the ovary is the ovarian cavity
  (locule)\\
  a) Statement A is correct but Statement B is incorrect\\
  b) Statement A is incorrect but Statement B is correct\\
  c) Both statements are correct\\
  d) Both statements are incorrect
\item
  Identify the incorrect statement related to pollination\\
  a) Pollination by water is quite rare in flowering plants\\
  b) Pollination by wind is more common amongst abiotic pollination\\
  c) Flowers produce foul odours to attract flies and beetles to get
  pollinated\\
  d) Moths and butterflies are the most dominant pollinating agents
  among insects
\item
  \textbf{Assertion (A):} Megaspore mother cell undergoes meiosis to
  produce four haploid gametes of which all are functional\\
  \textbf{Reason (R):} Megaspore mother cell is 2n, mitosis gives
  haploid structure.
\end{enumerate}

a) Assertion and reason are true and reason is the correct explanation
of assertion

b) Assertion and reason are true but reason is not the correct
explanation of assertion

c) Assertion is true but reason is false

d) Both assertion and reason are false

\begin{enumerate}
\def\labelenumi{\arabic{enumi}.}
\setcounter{enumi}{147}
\item
  A typical angiosperm embryo sac at maturity is\\
  a) 8-nucleate and 8-celled b) 8-nucleate and 7-celled\\
  c) 7-nucleate and 8-celled d) 7-nucleate and 7-celled
\item
  \textbf{Statement - A:} Some examples of water pollinated plants are
  \emph{Vallisneria} and \emph{Hydrilla} which grow in marine water and
  several fresh water species such as \emph{Zostera}.\\
  \textbf{Statement -B :} The synergids have special cellular
  thickenings at the chalazal tip called filiform apparatus, which play
  an important role in guiding the pollen tubes into the synergid.\\
  a) Statement A is correct but Statement B is incorrect\\
  b) Statement A is incorrect but Statement B is correct\\
  c) Both statements are correct\\
  d) Both statements are incorrect
\item
  In some members of which of the following pairs of families, pollen
  grains retain their viability for months after release?\\
  a) Rosaceae, Leguminosae b) Poaceae, Rosaceae\\
  c) Poaceae, Leguminosae d) Poaceae, Solanaceae
\item
  \textbf{Assertion (A):} Flowers are the structures related to sexual
  reproduction in flowering plants\\
  \textbf{Reason (R):} Various embryological processes of plants occur
  in a flower\\
  a) Assertion and reason are true and reason is the correct explanation
  of assertion
\end{enumerate}

b) Assertion and reason are true but reason is not the correct
explanation of assertion

c) Assertion is true but reason is false

d) Assertion is false but reason is true

\begin{enumerate}
\def\labelenumi{\arabic{enumi}.}
\setcounter{enumi}{151}
\item
  The plant parts which consist of two generations one within the
  other\\
  a) pollen grains inside the megaspores\\
  b) germinated pollen grain with two male gametes\\
  c) seed inside the fruit\\
  d) embryo sac inside the ovule
\item
  \textbf{Statement-A:} Among the animals, insects, particularly bees
  are the dominant biotic pollinating agents\\
  \textbf{Statement-B:} Continued self-pollination result in inbreeding
  depression\\
  a) Statement A is correct but Statement B is incorrect\\
  b) Statement A is incorrect but Statement B is correct\\
  c) Both statements are correct\\
  d) Both statements are incorrect
\item
  Which is the most common type of embryo sac in angiosperms?\\
  a) Tetrasporic with one mitotic stage of divisions\\
  b) Monosporic with three sequential mitotic divisions\\
  c) Monosporic with two sequential miotic divisions\\
  d) Bisporic with two sequential mitotic divisions
\item
  Pollen grains can be stored for several years in liquid nitrogen
  having a temperature of\\
  a) -120°C b) -80°C c) -196°C d) -160°C
\item
  Which of the following has proved helpful in preserving pollen as
  fossils?\\
  a) Pollenkitt b) Cellulosic intine c) Oil content d) Sporopollenin
\item
  Functional megaspore in an angiosperm develops into an\\
  a) endosperm b) embryo sac c) embryo d) ovule
\item
  The developed embryo sac contains the central cell, which is
\end{enumerate}

a) Single nucleate b) Binucleate c) Four nucleate d) Eight nucleate

\begin{enumerate}
\def\labelenumi{\arabic{enumi}.}
\setcounter{enumi}{158}
\item
  What does filiform apparatus do at the entrance into ovule.?\\
  a) Brings about opening of pollen tube\\
  b) Guides pollen tube from synergid to egg\\
  c) Helps in entry of pollen tube into synergid\\
  d) Prevents entry of more than one pollen tube into embryo sac
\item
  Ploidy of ovum of angiosperms is\\
  a) Haploid b) Diploid c) Triploid d) Polyploid
\item
  In a multicarpellary gynoecium, the pistils may be fused together in
  case of \_\_\_ gynoecium, or free in case of \_\_\_\_ gynoecium.\\
  a) Apocarpous, syncarpous b) Syncarpous, apocarpous\\
  c) Syncarpous, syncarpous d) apocarpous, apocarpous
\item
  What is functional megaspore referred to as?\\
  a) The megaspore that degenerates after formation\\
  b) The megaspore that does not degenerate and undergoes mitosis
  later\\
  c) The megaspore that undergoes reduction division\\
  d) The megaspore that is functionally inactive
\item
  Which of the following nucleus is unlike other nuclei in the female
  gametophyte of angiosperms?\\
  a) Egg nucleus b) Nucleus of synergids\\
  c) Secondary nucleus d) Nuclei of antipodals
\item
  Ovule of most angiosperms is\\
  a) Orthotropus b) Anatropous c) Campylotropus d) Amphitropous
\item
  The tips on the ovule where integument are absent are called\\
  a) Micropyle b) Germ pore c) Both a) and b) d) Integuments
\item
  Out of the 4 megaspores, how many functional megaspores would
  remain?\\
  a) 2 b) 1 c) 3 d) 2
\item
  Antipodal cells are present in which end of the embryo sac?\\
  a) Chalazal end b) Micropylar end\\
  c) They are situated in the centre d) None of the above
\item
  Depending on the source of pollen, pollination can be divided into\\
  a) Two types b) Three types\\
  c) Four types d) Many types
\item
  The conditions required for the autogamy\\
  a) Bisexuality\\
  b) Synchrony in pollen release and stigma receptivity\\
  c) Anther and stigma should lie close to each other\\
  d) All of the above
\item
  In the corn cob, the tassels which wave in the wind to trap the pollen
  grains represents\\
  a) Stigma and style b) Style and ovary c) Stigma d) Style
\item
  In most of the water pollinated species, pollen grains are protected
  from wetting by a\\
  a) Mucilaginous covering b) Agar coating\\
  c) Algin coating d) Pectose coating
\item
  Which of the following species provides floral rewards in the form of
  providing safe place to lay eggs?\\
  a) Amorphophallus b) Fig c) Yucca d) All of the above
\item
  Yucca plant is pollinated by\\
  a) A species of moth (\emph{Pronuba}) b) A species of wasp
  (\emph{Blastophaga})\\
  c) A species of beetle d) A species of insect
\item
  Cleistogamous flower is found in\\
  a) Tobacco b) \emph{Mirabilis} c) \emph{Viola} d) None of the above
\item
  Contrivance for self - pollination is\\
  a) Cleistogamy b) Bisexuality\\
  c) Homogamy d) All of the above
\item
  Having flowers that are unisexual inhibits\\
  a) Geitonogamy but not xenogamy b) Autogamy and geitonogamy\\
  c) Autogamy but not geitonogamy d) Both geitonogamy and xenogamy
\item
  The kind of pollination that transports pollen grains of genetically
  distinct types to the stigma of a plant is referred to as\\
  a) Xenogamy b) Geitonogamy c) Chasmogamy d) Autogamy
\item
  In which category, pollination is invariably autogamous?\\
  a) Chasmogamy b) Geitonogamy c) Cleistogamy d) Xenogamy
\item
  Fragrant flowers with well -- developed nectaries are an adaptation
  for\\
  a) Zoophily b) Anemophily c) Entomophily d) Hydrophily
\item
  Both autogamy and geitonogamy cannot occur in\\
  a) Papaya b) Cucumber c) Castor d) Maize
\end{enumerate}
